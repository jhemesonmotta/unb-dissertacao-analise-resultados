% Options for packages loaded elsewhere
\PassOptionsToPackage{unicode}{hyperref}
\PassOptionsToPackage{hyphens}{url}
%
\documentclass[
]{article}
\usepackage{lmodern}
\usepackage{amssymb,amsmath}
\usepackage{ifxetex,ifluatex}
\ifnum 0\ifxetex 1\fi\ifluatex 1\fi=0 % if pdftex
  \usepackage[T1]{fontenc}
  \usepackage[utf8]{inputenc}
  \usepackage{textcomp} % provide euro and other symbols
\else % if luatex or xetex
  \usepackage{unicode-math}
  \defaultfontfeatures{Scale=MatchLowercase}
  \defaultfontfeatures[\rmfamily]{Ligatures=TeX,Scale=1}
\fi
% Use upquote if available, for straight quotes in verbatim environments
\IfFileExists{upquote.sty}{\usepackage{upquote}}{}
\IfFileExists{microtype.sty}{% use microtype if available
  \usepackage[]{microtype}
  \UseMicrotypeSet[protrusion]{basicmath} % disable protrusion for tt fonts
}{}
\makeatletter
\@ifundefined{KOMAClassName}{% if non-KOMA class
  \IfFileExists{parskip.sty}{%
    \usepackage{parskip}
  }{% else
    \setlength{\parindent}{0pt}
    \setlength{\parskip}{6pt plus 2pt minus 1pt}}
}{% if KOMA class
  \KOMAoptions{parskip=half}}
\makeatother
\usepackage{xcolor}
\IfFileExists{xurl.sty}{\usepackage{xurl}}{} % add URL line breaks if available
\IfFileExists{bookmark.sty}{\usepackage{bookmark}}{\usepackage{hyperref}}
\hypersetup{
  pdftitle={Obversações GSOC},
  pdfauthor={blind},
  hidelinks,
  pdfcreator={LaTeX via pandoc}}
\urlstyle{same} % disable monospaced font for URLs
\usepackage[margin=1in]{geometry}
\usepackage{color}
\usepackage{fancyvrb}
\newcommand{\VerbBar}{|}
\newcommand{\VERB}{\Verb[commandchars=\\\{\}]}
\DefineVerbatimEnvironment{Highlighting}{Verbatim}{commandchars=\\\{\}}
% Add ',fontsize=\small' for more characters per line
\usepackage{framed}
\definecolor{shadecolor}{RGB}{248,248,248}
\newenvironment{Shaded}{\begin{snugshade}}{\end{snugshade}}
\newcommand{\AlertTok}[1]{\textcolor[rgb]{0.94,0.16,0.16}{#1}}
\newcommand{\AnnotationTok}[1]{\textcolor[rgb]{0.56,0.35,0.01}{\textbf{\textit{#1}}}}
\newcommand{\AttributeTok}[1]{\textcolor[rgb]{0.77,0.63,0.00}{#1}}
\newcommand{\BaseNTok}[1]{\textcolor[rgb]{0.00,0.00,0.81}{#1}}
\newcommand{\BuiltInTok}[1]{#1}
\newcommand{\CharTok}[1]{\textcolor[rgb]{0.31,0.60,0.02}{#1}}
\newcommand{\CommentTok}[1]{\textcolor[rgb]{0.56,0.35,0.01}{\textit{#1}}}
\newcommand{\CommentVarTok}[1]{\textcolor[rgb]{0.56,0.35,0.01}{\textbf{\textit{#1}}}}
\newcommand{\ConstantTok}[1]{\textcolor[rgb]{0.00,0.00,0.00}{#1}}
\newcommand{\ControlFlowTok}[1]{\textcolor[rgb]{0.13,0.29,0.53}{\textbf{#1}}}
\newcommand{\DataTypeTok}[1]{\textcolor[rgb]{0.13,0.29,0.53}{#1}}
\newcommand{\DecValTok}[1]{\textcolor[rgb]{0.00,0.00,0.81}{#1}}
\newcommand{\DocumentationTok}[1]{\textcolor[rgb]{0.56,0.35,0.01}{\textbf{\textit{#1}}}}
\newcommand{\ErrorTok}[1]{\textcolor[rgb]{0.64,0.00,0.00}{\textbf{#1}}}
\newcommand{\ExtensionTok}[1]{#1}
\newcommand{\FloatTok}[1]{\textcolor[rgb]{0.00,0.00,0.81}{#1}}
\newcommand{\FunctionTok}[1]{\textcolor[rgb]{0.00,0.00,0.00}{#1}}
\newcommand{\ImportTok}[1]{#1}
\newcommand{\InformationTok}[1]{\textcolor[rgb]{0.56,0.35,0.01}{\textbf{\textit{#1}}}}
\newcommand{\KeywordTok}[1]{\textcolor[rgb]{0.13,0.29,0.53}{\textbf{#1}}}
\newcommand{\NormalTok}[1]{#1}
\newcommand{\OperatorTok}[1]{\textcolor[rgb]{0.81,0.36,0.00}{\textbf{#1}}}
\newcommand{\OtherTok}[1]{\textcolor[rgb]{0.56,0.35,0.01}{#1}}
\newcommand{\PreprocessorTok}[1]{\textcolor[rgb]{0.56,0.35,0.01}{\textit{#1}}}
\newcommand{\RegionMarkerTok}[1]{#1}
\newcommand{\SpecialCharTok}[1]{\textcolor[rgb]{0.00,0.00,0.00}{#1}}
\newcommand{\SpecialStringTok}[1]{\textcolor[rgb]{0.31,0.60,0.02}{#1}}
\newcommand{\StringTok}[1]{\textcolor[rgb]{0.31,0.60,0.02}{#1}}
\newcommand{\VariableTok}[1]{\textcolor[rgb]{0.00,0.00,0.00}{#1}}
\newcommand{\VerbatimStringTok}[1]{\textcolor[rgb]{0.31,0.60,0.02}{#1}}
\newcommand{\WarningTok}[1]{\textcolor[rgb]{0.56,0.35,0.01}{\textbf{\textit{#1}}}}
\usepackage{graphicx,grffile}
\makeatletter
\def\maxwidth{\ifdim\Gin@nat@width>\linewidth\linewidth\else\Gin@nat@width\fi}
\def\maxheight{\ifdim\Gin@nat@height>\textheight\textheight\else\Gin@nat@height\fi}
\makeatother
% Scale images if necessary, so that they will not overflow the page
% margins by default, and it is still possible to overwrite the defaults
% using explicit options in \includegraphics[width, height, ...]{}
\setkeys{Gin}{width=\maxwidth,height=\maxheight,keepaspectratio}
% Set default figure placement to htbp
\makeatletter
\def\fps@figure{htbp}
\makeatother
\setlength{\emergencystretch}{3em} % prevent overfull lines
\providecommand{\tightlist}{%
  \setlength{\itemsep}{0pt}\setlength{\parskip}{0pt}}
\setcounter{secnumdepth}{-\maxdimen} % remove section numbering

\title{Obversações GSOC}
\author{blind}
\date{18 de Setembro de 2019}

\begin{document}
\maketitle

\begin{center}\rule{0.5\linewidth}{0.5pt}\end{center}

\textbf{Carrega os dados do CSV para o DataFrame:}

\begin{Shaded}
\begin{Highlighting}[]
\NormalTok{gsoc_data <-}\StringTok{ }\KeywordTok{read.csv}\NormalTok{(}\DataTypeTok{file=}\StringTok{"gsoc_data.csv"}\NormalTok{, }\DataTypeTok{header=}\OtherTok{TRUE}\NormalTok{, }\DataTypeTok{sep=}\StringTok{","}\NormalTok{)}
\NormalTok{myPalette <-}\StringTok{ }\KeywordTok{brewer.pal}\NormalTok{(}\DecValTok{5}\NormalTok{, }\StringTok{"Set2"}\NormalTok{) }
\end{Highlighting}
\end{Shaded}

\hypertarget{gruxe1ficos}{%
\subsection{Gráficos:}\label{gruxe1ficos}}

\hypertarget{quantidade-de-participantes-por-ano-com-diferenciauxe7uxe3o-de-sexo}{%
\paragraph{Quantidade de Participantes por Ano com Diferenciação de
sexo:}\label{quantidade-de-participantes-por-ano-com-diferenciauxe7uxe3o-de-sexo}}

\begin{Shaded}
\begin{Highlighting}[]
\KeywordTok{qplot}\NormalTok{(}\DataTypeTok{x =}\NormalTok{ Year, }\DataTypeTok{fill =}\NormalTok{ Gender, }\DataTypeTok{data =}\NormalTok{ gsoc_data, }\DataTypeTok{binwidth =} \DecValTok{1}\NormalTok{, }\DataTypeTok{xlab=}\StringTok{"Year"}\NormalTok{, }\DataTypeTok{ylab=}\StringTok{"Number of People"}\NormalTok{)}
\end{Highlighting}
\end{Shaded}

\includegraphics{graficos_files/figure-latex/unnamed-chunk-2-1.pdf}

\hypertarget{quantidade-de-participantes-do-sexo-feminino-por-ano}{%
\paragraph{Quantidade de Participantes do Sexo Feminino por
Ano:}\label{quantidade-de-participantes-do-sexo-feminino-por-ano}}

Selecionar apenas mulheres.

\begin{Shaded}
\begin{Highlighting}[]
\NormalTok{mulheres <-}\StringTok{ }\NormalTok{gsoc_data }\OperatorTok
\StringTok{  }\KeywordTok{filter}\NormalTok{(Gender }\OperatorTok{==}\StringTok{ "female"}\NormalTok{)}
\end{Highlighting}
\end{Shaded}

\begin{Shaded}
\begin{Highlighting}[]
\KeywordTok{qplot}\NormalTok{(}\DataTypeTok{x =}\NormalTok{ Year, }\DataTypeTok{data =}\NormalTok{ mulheres, }\DataTypeTok{binwidth =} \DecValTok{1}\NormalTok{, }\DataTypeTok{xlab=}\StringTok{"Year"}\NormalTok{, }\DataTypeTok{ylab=}\StringTok{"Number of Women"}\NormalTok{) }
\end{Highlighting}
\end{Shaded}

\includegraphics{graficos_files/figure-latex/unnamed-chunk-4-1.pdf}

\hypertarget{quantidade-de-mentores-do-sexo-feminino-por-ano}{%
\paragraph{Quantidade de Mentores do Sexo Feminino por
Ano:}\label{quantidade-de-mentores-do-sexo-feminino-por-ano}}

\begin{Shaded}
\begin{Highlighting}[]
\NormalTok{mulheres <-}\StringTok{ }\NormalTok{gsoc_data }\OperatorTok
\StringTok{  }\KeywordTok{filter}\NormalTok{(Gender }\OperatorTok{==}\StringTok{ "female"} \OperatorTok{&}\StringTok{ }\NormalTok{Person.Type }\OperatorTok{==}\StringTok{ "mentor"}\NormalTok{)}
\end{Highlighting}
\end{Shaded}

\begin{Shaded}
\begin{Highlighting}[]
\KeywordTok{qplot}\NormalTok{(}\DataTypeTok{x =}\NormalTok{ Year, }\DataTypeTok{data =}\NormalTok{ mulheres, }\DataTypeTok{binwidth =} \DecValTok{1}\NormalTok{, }\DataTypeTok{xlab=}\StringTok{"Year"}\NormalTok{, }\DataTypeTok{ylab=}\StringTok{"Number of Female Mentors"}\NormalTok{)}
\end{Highlighting}
\end{Shaded}

\includegraphics{graficos_files/figure-latex/unnamed-chunk-6-1.pdf}

\hypertarget{quantidade-de-estudantes-do-sexo-feminino-por-ano}{%
\paragraph{Quantidade de Estudantes do Sexo Feminino por
Ano:}\label{quantidade-de-estudantes-do-sexo-feminino-por-ano}}

\begin{Shaded}
\begin{Highlighting}[]
\NormalTok{mulheres <-}\StringTok{ }\NormalTok{gsoc_data }\OperatorTok
\StringTok{  }\KeywordTok{filter}\NormalTok{(Gender }\OperatorTok{==}\StringTok{ "female"} \OperatorTok{&}\StringTok{ }\NormalTok{Person.Type }\OperatorTok{==}\StringTok{ "student"}\NormalTok{)}
\end{Highlighting}
\end{Shaded}

\begin{Shaded}
\begin{Highlighting}[]
\KeywordTok{qplot}\NormalTok{(}\DataTypeTok{x =}\NormalTok{ Year, }\DataTypeTok{data =}\NormalTok{ mulheres, }\DataTypeTok{binwidth =} \DecValTok{1}\NormalTok{, }\DataTypeTok{xlab=}\StringTok{"Year"}\NormalTok{, }\DataTypeTok{ylab=}\StringTok{"Number of Female Students"}\NormalTok{)}
\end{Highlighting}
\end{Shaded}

\includegraphics{graficos_files/figure-latex/unnamed-chunk-8-1.pdf}

\hypertarget{porcentagem-de-participantes-2016}{%
\paragraph{Porcentagem de participantes
2016:}\label{porcentagem-de-participantes-2016}}

\begin{Shaded}
\begin{Highlighting}[]
\NormalTok{pctMulheres1 =}\StringTok{ }\NormalTok{(}\KeywordTok{nrow}\NormalTok{(gsoc_data }\OperatorTok\StringTok{ }\KeywordTok{filter}\NormalTok{(Year }\OperatorTok{==}\StringTok{ "2016"} \OperatorTok{&}\StringTok{ }\NormalTok{Gender}\OperatorTok{==}\StringTok{"female"}\NormalTok{)) }\OperatorTok{/}\StringTok{ }\KeywordTok{nrow}\NormalTok{(gsoc_data }\OperatorTok\StringTok{ }\KeywordTok{filter}\NormalTok{(Year }\OperatorTok{==}\StringTok{ "2016"}\NormalTok{))) }\OperatorTok{*}\StringTok{ }\DecValTok{100}

\NormalTok{pctMulheres1}
\end{Highlighting}
\end{Shaded}

\begin{verbatim}
## [1] 42.16618
\end{verbatim}

\begin{Shaded}
\begin{Highlighting}[]
\NormalTok{pctHomens =}\StringTok{ }\NormalTok{(}\KeywordTok{nrow}\NormalTok{(gsoc_data }\OperatorTok\StringTok{ }\KeywordTok{filter}\NormalTok{(Year }\OperatorTok{==}\StringTok{ "2016"} \OperatorTok{&}\StringTok{ }\NormalTok{Gender}\OperatorTok{==}\StringTok{"male"}\NormalTok{)) }\OperatorTok{/}\StringTok{ }\KeywordTok{nrow}\NormalTok{(gsoc_data }\OperatorTok\StringTok{ }\KeywordTok{filter}\NormalTok{(Year }\OperatorTok{==}\StringTok{ "2016"}\NormalTok{))) }\OperatorTok{*}\StringTok{ }\DecValTok{100}

\NormalTok{pctHomens}
\end{Highlighting}
\end{Shaded}

\begin{verbatim}
## [1] 54.94006
\end{verbatim}

\begin{Shaded}
\begin{Highlighting}[]
\NormalTok{listavalores <-}\StringTok{ }\KeywordTok{c}\NormalTok{(pctHomens, pctMulheres1)}
\NormalTok{listalabels <-}\StringTok{ }\KeywordTok{c}\NormalTok{(}\StringTok{"Men"}\NormalTok{, }\StringTok{"Women"}\NormalTok{)}

\KeywordTok{pie}\NormalTok{(listavalores, listalabels, }\DataTypeTok{border=}\StringTok{"white"}\NormalTok{, }\DataTypeTok{col=}\NormalTok{myPalette)}
\end{Highlighting}
\end{Shaded}

\includegraphics{graficos_files/figure-latex/unnamed-chunk-11-1.pdf}

\hypertarget{porcentagem-de-participantes-2017}{%
\paragraph{Porcentagem de participantes
2017:}\label{porcentagem-de-participantes-2017}}

\begin{Shaded}
\begin{Highlighting}[]
\NormalTok{pctMulheres2 =}\StringTok{ }\NormalTok{(}\KeywordTok{nrow}\NormalTok{(gsoc_data }\OperatorTok\StringTok{ }\KeywordTok{filter}\NormalTok{(Year }\OperatorTok{==}\StringTok{ "2017"} \OperatorTok{&}\StringTok{ }\NormalTok{Gender}\OperatorTok{==}\StringTok{"female"}\NormalTok{)) }\OperatorTok{/}\StringTok{ }\KeywordTok{nrow}\NormalTok{(gsoc_data }\OperatorTok\StringTok{ }\KeywordTok{filter}\NormalTok{(Year }\OperatorTok{==}\StringTok{ "2017"}\NormalTok{))) }\OperatorTok{*}\StringTok{ }\DecValTok{100}

\NormalTok{pctMulheres2}
\end{Highlighting}
\end{Shaded}

\begin{verbatim}
## [1] 22.55821
\end{verbatim}

\begin{Shaded}
\begin{Highlighting}[]
\NormalTok{pctHomens =}\StringTok{ }\NormalTok{(}\KeywordTok{nrow}\NormalTok{(gsoc_data }\OperatorTok\StringTok{ }\KeywordTok{filter}\NormalTok{(Year }\OperatorTok{==}\StringTok{ "2017"} \OperatorTok{&}\StringTok{ }\NormalTok{Gender}\OperatorTok{==}\StringTok{"male"}\NormalTok{)) }\OperatorTok{/}\StringTok{ }\KeywordTok{nrow}\NormalTok{(gsoc_data }\OperatorTok\StringTok{ }\KeywordTok{filter}\NormalTok{(Year }\OperatorTok{==}\StringTok{ "2017"}\NormalTok{))) }\OperatorTok{*}\StringTok{ }\DecValTok{100}

\NormalTok{pctHomens}
\end{Highlighting}
\end{Shaded}

\begin{verbatim}
## [1] 72.785
\end{verbatim}

\begin{Shaded}
\begin{Highlighting}[]
\NormalTok{listavalores <-}\StringTok{ }\KeywordTok{c}\NormalTok{(pctHomens, pctMulheres2)}
\NormalTok{listalabels <-}\StringTok{ }\KeywordTok{c}\NormalTok{(}\StringTok{"Men"}\NormalTok{, }\StringTok{"Women"}\NormalTok{)}

\KeywordTok{pie}\NormalTok{(listavalores, listalabels, }\DataTypeTok{border=}\StringTok{"white"}\NormalTok{, }\DataTypeTok{col=}\NormalTok{myPalette)}
\end{Highlighting}
\end{Shaded}

\includegraphics{graficos_files/figure-latex/unnamed-chunk-14-1.pdf}

\hypertarget{porcentagem-de-participantes-2018}{%
\paragraph{Porcentagem de participantes
2018:}\label{porcentagem-de-participantes-2018}}

\begin{Shaded}
\begin{Highlighting}[]
\NormalTok{pctMulheres3 =}\StringTok{ }\NormalTok{(}\KeywordTok{nrow}\NormalTok{(gsoc_data }\OperatorTok\StringTok{ }\KeywordTok{filter}\NormalTok{(Year }\OperatorTok{==}\StringTok{ "2018"} \OperatorTok{&}\StringTok{ }\NormalTok{Gender}\OperatorTok{==}\StringTok{"female"}\NormalTok{)) }\OperatorTok{/}\StringTok{ }\KeywordTok{nrow}\NormalTok{(gsoc_data }\OperatorTok\StringTok{ }\KeywordTok{filter}\NormalTok{(Year }\OperatorTok{==}\StringTok{ "2018"}\NormalTok{))) }\OperatorTok{*}\StringTok{ }\DecValTok{100}

\NormalTok{pctMulheres3}
\end{Highlighting}
\end{Shaded}

\begin{verbatim}
## [1] 21.77864
\end{verbatim}

\begin{Shaded}
\begin{Highlighting}[]
\NormalTok{pctHomens =}\StringTok{ }\NormalTok{(}\KeywordTok{nrow}\NormalTok{(gsoc_data }\OperatorTok\StringTok{ }\KeywordTok{filter}\NormalTok{(Year }\OperatorTok{==}\StringTok{ "2018"} \OperatorTok{&}\StringTok{ }\NormalTok{Gender}\OperatorTok{==}\StringTok{"male"}\NormalTok{)) }\OperatorTok{/}\StringTok{ }\KeywordTok{nrow}\NormalTok{(gsoc_data }\OperatorTok\StringTok{ }\KeywordTok{filter}\NormalTok{(Year }\OperatorTok{==}\StringTok{ "2018"}\NormalTok{))) }\OperatorTok{*}\StringTok{ }\DecValTok{100}

\NormalTok{pctHomens}
\end{Highlighting}
\end{Shaded}

\begin{verbatim}
## [1] 72.8335
\end{verbatim}

\begin{Shaded}
\begin{Highlighting}[]
\NormalTok{listavalores <-}\StringTok{ }\KeywordTok{c}\NormalTok{(pctHomens, pctMulheres3)}
\NormalTok{listalabels <-}\StringTok{ }\KeywordTok{c}\NormalTok{(}\StringTok{"Men"}\NormalTok{, }\StringTok{"Women"}\NormalTok{)}


\KeywordTok{pie}\NormalTok{(listavalores, listalabels, }\DataTypeTok{border=}\StringTok{"white"}\NormalTok{, }\DataTypeTok{col=}\NormalTok{myPalette)}
\end{Highlighting}
\end{Shaded}

\includegraphics{graficos_files/figure-latex/unnamed-chunk-17-1.pdf}

\hypertarget{porcentagem-de-participantes-2019}{%
\paragraph{Porcentagem de participantes
2019:}\label{porcentagem-de-participantes-2019}}

\begin{Shaded}
\begin{Highlighting}[]
\NormalTok{pctMulheres4 =}\StringTok{ }\NormalTok{(}\KeywordTok{nrow}\NormalTok{(gsoc_data }\OperatorTok\StringTok{ }\KeywordTok{filter}\NormalTok{(Year }\OperatorTok{==}\StringTok{ "2019"} \OperatorTok{&}\StringTok{ }\NormalTok{Gender}\OperatorTok{==}\StringTok{"female"}\NormalTok{)) }\OperatorTok{/}\StringTok{ }\KeywordTok{nrow}\NormalTok{(gsoc_data }\OperatorTok\StringTok{ }\KeywordTok{filter}\NormalTok{(Year }\OperatorTok{==}\StringTok{ "2019"}\NormalTok{))) }\OperatorTok{*}\StringTok{ }\DecValTok{100}

\NormalTok{pctMulheres4}
\end{Highlighting}
\end{Shaded}

\begin{verbatim}
## [1] 22.04209
\end{verbatim}

\begin{Shaded}
\begin{Highlighting}[]
\NormalTok{pctHomens =}\StringTok{ }\NormalTok{(}\KeywordTok{nrow}\NormalTok{(gsoc_data }\OperatorTok\StringTok{ }\KeywordTok{filter}\NormalTok{(Year }\OperatorTok{==}\StringTok{ "2019"} \OperatorTok{&}\StringTok{ }\NormalTok{Gender}\OperatorTok{==}\StringTok{"male"}\NormalTok{)) }\OperatorTok{/}\StringTok{ }\KeywordTok{nrow}\NormalTok{(gsoc_data }\OperatorTok\StringTok{ }\KeywordTok{filter}\NormalTok{(Year }\OperatorTok{==}\StringTok{ "2019"}\NormalTok{))) }\OperatorTok{*}\StringTok{ }\DecValTok{100}

\NormalTok{pctHomens}
\end{Highlighting}
\end{Shaded}

\begin{verbatim}
## [1] 72.32651
\end{verbatim}

\begin{Shaded}
\begin{Highlighting}[]
\NormalTok{listavalores <-}\StringTok{ }\KeywordTok{c}\NormalTok{(pctHomens, pctMulheres3)}
\NormalTok{listalabels <-}\StringTok{ }\KeywordTok{c}\NormalTok{(}\StringTok{"Men"}\NormalTok{, }\StringTok{"Women"}\NormalTok{)}


\KeywordTok{pie}\NormalTok{(listavalores, listalabels, }\DataTypeTok{border=}\StringTok{"white"}\NormalTok{, }\DataTypeTok{col=}\NormalTok{myPalette)}
\end{Highlighting}
\end{Shaded}

\includegraphics{graficos_files/figure-latex/unnamed-chunk-20-1.pdf}

\hypertarget{porcentagem-de-mulheres-por-ano}{%
\paragraph{Porcentagem de mulheres por
ano}\label{porcentagem-de-mulheres-por-ano}}

\begin{Shaded}
\begin{Highlighting}[]
\NormalTok{porcentagemMulheres <-}\StringTok{ }\KeywordTok{c}\NormalTok{(pctMulheres1,pctMulheres2,pctMulheres3,pctMulheres4)}
\NormalTok{anos <-}\StringTok{ }\KeywordTok{c}\NormalTok{(}\DecValTok{2016}\NormalTok{ , }\DecValTok{2017}\NormalTok{, }\DecValTok{2018}\NormalTok{, }\DecValTok{2019}\NormalTok{)}

\NormalTok{porcentagemMulheresPorAno <-}\StringTok{ }\KeywordTok{data.frame}\NormalTok{(porcentagemMulheres,anos)}

\KeywordTok{qplot}\NormalTok{(}\DataTypeTok{x =}\NormalTok{ anos, }\DataTypeTok{y =}\NormalTok{ porcentagemMulheres, }\DataTypeTok{data =}\NormalTok{ porcentagemMulheresPorAno, }\DataTypeTok{geom =} \StringTok{"line"}\NormalTok{, }\DataTypeTok{xlab=}\StringTok{"Year"}\NormalTok{, }\DataTypeTok{ylab=}\StringTok{"Women Percentage"}\NormalTok{)}
\end{Highlighting}
\end{Shaded}

\includegraphics{graficos_files/figure-latex/unnamed-chunk-21-1.pdf}

\hypertarget{comparauxe7uxe3o-de-mentores-e-estudantes}{%
\paragraph{Comparação de Mentores e
Estudantes}\label{comparauxe7uxe3o-de-mentores-e-estudantes}}

\begin{Shaded}
\begin{Highlighting}[]
\NormalTok{qtdMentores =}\StringTok{ }\NormalTok{(}\KeywordTok{nrow}\NormalTok{(gsoc_data }\OperatorTok\StringTok{ }\KeywordTok{filter}\NormalTok{(Person.Type }\OperatorTok{==}\StringTok{ "mentor"}\NormalTok{)))}
\NormalTok{qtdEstudantes =}\StringTok{ }\NormalTok{(}\KeywordTok{nrow}\NormalTok{(gsoc_data }\OperatorTok\StringTok{ }\KeywordTok{filter}\NormalTok{(Person.Type }\OperatorTok{==}\StringTok{ "student"}\NormalTok{)))}

\NormalTok{qtdMentores}
\end{Highlighting}
\end{Shaded}

\begin{verbatim}
## [1] 8379
\end{verbatim}

\begin{Shaded}
\begin{Highlighting}[]
\NormalTok{qtdEstudantes}
\end{Highlighting}
\end{Shaded}

\begin{verbatim}
## [1] 3944
\end{verbatim}

\begin{Shaded}
\begin{Highlighting}[]
\NormalTok{listavalores <-}\StringTok{ }\KeywordTok{c}\NormalTok{(qtdMentores, qtdEstudantes)}
\NormalTok{listalabels <-}\StringTok{ }\KeywordTok{c}\NormalTok{(}\StringTok{"Mentors"}\NormalTok{, }\StringTok{"Students"}\NormalTok{)}


\KeywordTok{pie}\NormalTok{(listavalores, listalabels, }\DataTypeTok{border=}\StringTok{"white"}\NormalTok{, }\DataTypeTok{col=}\NormalTok{myPalette)}
\end{Highlighting}
\end{Shaded}

\includegraphics{graficos_files/figure-latex/unnamed-chunk-23-1.pdf}
\#\#\#\# Comparação de Gêneros de Mentores

\begin{Shaded}
\begin{Highlighting}[]
\NormalTok{pctMulheresMentoras =}\StringTok{ }\NormalTok{(}\KeywordTok{nrow}\NormalTok{(gsoc_data }\OperatorTok\StringTok{ }\KeywordTok{filter}\NormalTok{(Person.Type }\OperatorTok{==}\StringTok{ "mentor"} \OperatorTok{&}\StringTok{ }\NormalTok{Gender}\OperatorTok{==}\StringTok{"female"}\NormalTok{)) }\OperatorTok{/}\StringTok{ }\KeywordTok{nrow}\NormalTok{(gsoc_data }\OperatorTok\StringTok{ }\KeywordTok{filter}\NormalTok{(Gender }\OperatorTok{==}\StringTok{ "female"}\NormalTok{))) }\OperatorTok{*}\StringTok{ }\DecValTok{100}

\NormalTok{pctHomensMentores =}\StringTok{ }\NormalTok{(}\KeywordTok{nrow}\NormalTok{(gsoc_data }\OperatorTok\StringTok{ }\KeywordTok{filter}\NormalTok{(Person.Type }\OperatorTok{==}\StringTok{ "mentor"} \OperatorTok{&}\StringTok{ }\NormalTok{Gender }\OperatorTok{==}\StringTok{ "male"}\NormalTok{)) }\OperatorTok{/}\StringTok{ }\KeywordTok{nrow}\NormalTok{(gsoc_data }\OperatorTok\StringTok{ }\KeywordTok{filter}\NormalTok{(Gender }\OperatorTok{==}\StringTok{ "male"}\NormalTok{))) }\OperatorTok{*}\StringTok{ }\DecValTok{100}


\NormalTok{pctMulheresMentoras}
\end{Highlighting}
\end{Shaded}

\begin{verbatim}
## [1] 69.33375
\end{verbatim}

\begin{Shaded}
\begin{Highlighting}[]
\NormalTok{pctHomensMentores}
\end{Highlighting}
\end{Shaded}

\begin{verbatim}
## [1] 67.62877
\end{verbatim}

\hypertarget{frequuxeancia-absoluta-das-tecnologias}{%
\paragraph{Frequência absoluta das
tecnologias}\label{frequuxeancia-absoluta-das-tecnologias}}

\begin{Shaded}
\begin{Highlighting}[]
\NormalTok{tecnologias.tb <-}\StringTok{ }\KeywordTok{table}\NormalTok{(gsoc_data}\OperatorTok{$}\NormalTok{Technology)}

\KeywordTok{barplot}\NormalTok{(tecnologias.tb, }\DataTypeTok{las=}\DecValTok{2}\NormalTok{)}
\end{Highlighting}
\end{Shaded}

\includegraphics{graficos_files/figure-latex/unnamed-chunk-25-1.pdf}

\hypertarget{gruxe1fico-das-5-tecnologias-mais-utilizadas}{%
\paragraph{gráfico das 5 tecnologias mais
utilizadas}\label{gruxe1fico-das-5-tecnologias-mais-utilizadas}}

\begin{Shaded}
\begin{Highlighting}[]
\NormalTok{tecnologias.tb <-}\StringTok{ }\KeywordTok{sort}\NormalTok{(tecnologias.tb)}

\KeywordTok{barplot}\NormalTok{(}\KeywordTok{tail}\NormalTok{(tecnologias.tb, }\DecValTok{5}\NormalTok{), }\DataTypeTok{las=}\DecValTok{2}\NormalTok{, }\DataTypeTok{cex.names=}\FloatTok{0.8}\NormalTok{)}
\end{Highlighting}
\end{Shaded}

\includegraphics{graficos_files/figure-latex/unnamed-chunk-26-1.pdf}

\hypertarget{frequuxeancia-absoluta-dos-tuxf3picos}{%
\paragraph{Frequência absoluta dos
tópicos}\label{frequuxeancia-absoluta-dos-tuxf3picos}}

\begin{Shaded}
\begin{Highlighting}[]
\NormalTok{topicos.tb <-}\StringTok{ }\KeywordTok{table}\NormalTok{(gsoc_data}\OperatorTok{$}\NormalTok{Topic)}
\KeywordTok{barplot}\NormalTok{(topicos.tb, }\DataTypeTok{las=}\DecValTok{2}\NormalTok{)}
\end{Highlighting}
\end{Shaded}

\includegraphics{graficos_files/figure-latex/unnamed-chunk-27-1.pdf}

\hypertarget{gruxe1fico-dos-20-tuxf3picos-mais-frequentes}{%
\paragraph{gráfico dos 20 tópicos mais
frequentes}\label{gruxe1fico-dos-20-tuxf3picos-mais-frequentes}}

\begin{Shaded}
\begin{Highlighting}[]
\NormalTok{topicos.tb <-}\StringTok{ }\KeywordTok{sort}\NormalTok{(topicos.tb)}

\KeywordTok{barplot}\NormalTok{(}\KeywordTok{tail}\NormalTok{(topicos.tb, }\DecValTok{20}\NormalTok{), }\DataTypeTok{las=}\DecValTok{2}\NormalTok{, }\DataTypeTok{cex.names=}\FloatTok{0.5}\NormalTok{)}
\end{Highlighting}
\end{Shaded}

\includegraphics{graficos_files/figure-latex/unnamed-chunk-28-1.pdf}

\hypertarget{gruxe1fico-dos-5-tuxf3picos-mais-frequentes}{%
\paragraph{gráfico dos 5 tópicos mais
frequentes}\label{gruxe1fico-dos-5-tuxf3picos-mais-frequentes}}

\begin{Shaded}
\begin{Highlighting}[]
\NormalTok{topicos.tb <-}\StringTok{ }\KeywordTok{sort}\NormalTok{(topicos.tb)}

\KeywordTok{barplot}\NormalTok{(}\KeywordTok{tail}\NormalTok{(topicos.tb, }\DecValTok{5}\NormalTok{), }\DataTypeTok{las=}\DecValTok{1}\NormalTok{, }\DataTypeTok{cex.names=}\FloatTok{0.8}\NormalTok{)}
\end{Highlighting}
\end{Shaded}

\includegraphics{graficos_files/figure-latex/unnamed-chunk-29-1.pdf}

\end{document}
